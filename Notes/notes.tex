\documentclass[a4paper,30pt]{report}

\usepackage{amsmath}
\usepackage{inputenc}
\usepackage[margin=2cm]{geometry}
\usepackage{graphicx}

\title{\Huge{\textbf{English Notes}}}
\author{MAHIB}

\begin{document}
  \maketitle\date{}\newpage
  \tableofcontents\newpage
  
  \chapter{Day 1 - 01/04/2024}
    Syllabus Overview done. Classroom Created. 
    
    \section{Cloze Test}
      In this test we have paragraphs with certain numbered blanks in them and we have some options for which word to fill in. We have to choose the most suitable option.\\

      \textbf{Example} -\\\\ \textbf{1.} Visual experiences can (1)children, teenagers and even
adults learn and absorb more due to its highly
stimulating and (2)engaging impact. It is for this
reason that we are seeing an increase in schools across
the globe (3) content provider programmes into their
class curriculum to (4) lessons through video. Visual
excursions and school collaborations are (5) by
advances in high definition video, high fidelity audio
and content sharing allowing students to experience a
richer and more stimulating learning experience.
Schools that have previously transported
students to excursions in (6), now face increased
transportation costs, higher insurance premiums,
attendance costs for the families and strict duty of care
policies for students while (7) school property/ Virtual
excursions (8) students to improve their presentation,
research, learning and speaking skills while they
engage in a live learning session. Students also now
have the ability to meet peers from many cultures,
speak to subject-matter (9) like scientists or authors
practise a foreign language with students from another
country, and learn about global issues from the (10) of
their own classrooms.\\\\

      \textbf{Options} -\\\\ 
1. (a) help (b) aiding
(c) prescribe (d) feature
(e) present\\\\
2. (a) plus (b) lonely
(c) ably (d) many
(e) deeply\\\\
3. (a) incorporating (b) pressing
(c) following (d) parting
(e) leaving\\\\
4. (a) make (b) demand
(c) impart (d) vision
(e) need\\\\
5. (a) dissolved (b) enhanced
(c) measured (d) failed
(e) blasted\\\\
6. (a) deed (b) total
(c) parent (d) person
(e) lieu\\\\
7. (a) involving (b) saving
(c) away (d) off
(e) vacating\\\\
8. (a) let (b) enable
(c) present (d) pressure
(e) collect\\\\
9. (a) clauses (b) dictionaries
(c) books (d) experts
(e) partners\\\\
10. (a) vacancy (b) availability
(c) safety (d) comfortable
(e) gap\\\\\\\\

     \textbf{Answers} - \\\\
1. (a) help\\\\
2. (a) make \\\\
3. (a) Incorporating\\\\
4. (c) Impart \\\\
5. (b) Enhanced \\\\
6. (d) Person \\\\
7. (a) Involving\\\\
8. (b) Enable \\\\
9. (d) Experts \\\\
10. (c) Safety \\\\

     \textbf{2.} If China’s state owned commercial banks seem
burdened by bad debts, the Country’s rural financial
sector is even worse. In the villages, the only formal
banking institutions are what are known as rural credit
co-operatives. These (11)the distinction in China of
having been officially declared insolvent. The rural
credit co-operatives are ill named. They are often
reluctant to(12) and they are not run as co-operatives
as they do not (13) any profits and their customers
have no say in their operations. Until 1996,they were
offshoots of the Agricultural Bank of China. Since
then they have been (14) by the Central Bank, though
they are in reality run by country government. Even
the word ‘rural’ is misleading. (15)of their deposits are
sucked up and put in the urban banking system.
Farmers usually find it easier to (16) from friends or

relatives or black market moneylenders. Yet the co-
operatives remain a big part of China’s financial

system. Last year, they (17) for 12 percent of deposits
and 11 percent of loans. In recent years, commercial
banks (including the Agricultural bank) have closed
down (18) in the countryside. Yet some 40,000 credit
co-operatives remain in place with one in almost every
township as the larger villages or smaller rural towns

are (19). If as the government claims, the credit co-
operatives are beginning to turn a profit after six years

of losses, it is not because they are any better run. In an
effort to (20) a stagnant rural economy, the central
bank has pumped more than \$9 billion into them
hoping that they will lend more to farmers. But the
root causes of their problems remain and the real
solution may have to involve a mix of approaches from
commercial banking to real cooperatives.\\\\

      \textbf{Options} - \\\\
11. (a) awarded (b) enjoy
(c) worry (d) making
(e) trouble\\\\
12. (a) sanctions (b) apply
(c) part (d) provide
(e) giving\\\\
13. (a) function (b) eligible
(c) claims (d) declared
(e) share\\\\
14. (a) own (b) govern
(c) regulations (d) ran
(e) supervised\\\\
15. (a) Such (b) Partly
(c) Whole (d) Most
(e) Entire\\\\
16.(a) visit (b) help
(c) borrow (d) loan
(e) advice\\\\
17.(a) include (b) accounted
(c) fulfilled (d) achieved
(e) taking\\\\
18.(a) branches (b) all
(c) operating (d) staff
(e) factory\\\\
19.(a) thinking (b)known
(c) creating (d) cross
(e) develop\\\\
20. (a) make (b)release
(c) boosting (d)stall
(e) revitalise\\\\

      \textbf{Answers} - \\\\
11. (b) Enjoy\\\\
12. (d) Provide\\\\
13. (e) share (?)\\\\
14. (e) Supervised\\\\
15. (d) Most\\\\
16. (c) Borrow\\\\
17. (b) Accounted\\\\
18. (a) Branches\\\\
19. (b) Known (?)\\\\
20. (e) Revitalise\\\\

  \chapter{Day 2 - }
 


\end{document}
