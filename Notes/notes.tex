\documentclass[a4paper,30pt]{report}

\usepackage{amsmath}
\usepackage{hyperref}
\usepackage{inputenc}
\usepackage[margin=2cm]{geometry}
\usepackage{graphicx}

\title{\Huge{\textbf{English Notes}}}
\author{MAHIB}

\begin{document}
  \maketitle\date{}\newpage
  \tableofcontents\newpage
  
  \chapter{Day 1 - 04/01/2024}
    Syllabus Overview done. Classroom Created. 
    
    \section{Cloze Test}
      In this test we have paragraphs with certain numbered blanks in them and we have some options for which word to fill in. We have to choose the most suitable option.\\

      \textbf{Example} -\\\\ \textbf{1.} Visual experiences can (1)children, teenagers and even
adults learn and absorb more due to its highly
stimulating and (2)engaging impact. It is for this
reason that we are seeing an increase in schools across
the globe (3) content provider programmes into their
class curriculum to (4) lessons through video. Visual
excursions and school collaborations are (5) by
advances in high definition video, high fidelity audio
and content sharing allowing students to experience a
richer and more stimulating learning experience.
Schools that have previously transported
students to excursions in (6), now face increased
transportation costs, higher insurance premiums,
attendance costs for the families and strict duty of care
policies for students while (7) school property/ Virtual
excursions (8) students to improve their presentation,
research, learning and speaking skills while they
engage in a live learning session. Students also now
have the ability to meet peers from many cultures,
speak to subject-matter (9) like scientists or authors
practise a foreign language with students from another
country, and learn about global issues from the (10) of
their own classrooms.\\\\

      \textbf{Options} -\\\\ 
1. (a) help (b) aiding
(c) prescribe (d) feature
(e) present\\\\
2. (a) plus (b) lonely
(c) ably (d) many
(e) deeply\\\\
3. (a) incorporating (b) pressing
(c) following (d) parting
(e) leaving\\\\
4. (a) make (b) demand
(c) impart (d) vision
(e) need\\\\
5. (a) dissolved (b) enhanced
(c) measured (d) failed
(e) blasted\\\\
6. (a) deed (b) total
(c) parent (d) person
(e) lieu\\\\
7. (a) involving (b) saving
(c) away (d) off
(e) vacating\\\\
8. (a) let (b) enable
(c) present (d) pressure
(e) collect\\\\
9. (a) clauses (b) dictionaries
(c) books (d) experts
(e) partners\\\\
10. (a) vacancy (b) availability
(c) safety (d) comfortable
(e) gap\\\\

     \textbf{Answers} - \\\\
1. (a) help\\\\
2. (a) make \\\\
3. (a) Incorporating\\\\
4. (c) Impart \\\\
5. (b) Enhanced \\\\
6. (d) Person \\\\
7. (a) Involving\\\\
8. (b) Enable \\\\
9. (d) Experts \\\\
10. (c) Safety \\\\

     \textbf{2.} If China’s state owned commercial banks seem
burdened by bad debts, the Country’s rural financial
sector is even worse. In the villages, the only formal
banking institutions are what are known as rural credit
co-operatives. These (11)the distinction in China of
having been officially declared insolvent. The rural
credit co-operatives are ill named. They are often
reluctant to(12) and they are not run as co-operatives
as they do not (13) any profits and their customers
have no say in their operations. Until 1996, they were
offshoots of the Agricultural Bank of China. Since
then they have been (14) by the Central Bank, though
they are in reality run by country government. Even
the word ‘rural’ is misleading. (15)of their deposits are
sucked up and put in the urban banking system.
Farmers usually find it easier to (16) from friends or

relatives or black market moneylenders. Yet the co-
operatives remain a big part of China’s financial

system. Last year, they (17) for 12 percent of deposits
and 11 percent of loans. In recent years, commercial
banks (including the Agricultural bank) have closed
down (18) in the countryside. Yet some 40,000 credit
co-operatives remain in place with one in almost every
township as the larger villages or smaller rural towns

are (19). If as the government claims, the credit co-
operatives are beginning to turn a profit after six years

of losses, it is not because they are any better run. In an
effort to (20) a stagnant rural economy, the central
bank has pumped more than \$9 billion into them
hoping that they will lend more to farmers. But the
root causes of their problems remain and the real
solution may have to involve a mix of approaches from
commercial banking to real cooperatives.\\\\

      \textbf{Options} - \\\\
11. (a) awarded (b) enjoy
(c) worry (d) making
(e) trouble\\\\
12. (a) sanctions (b) apply
(c) part (d) provide
(e) giving\\\\
13. (a) function (b) eligible
(c) claims (d) declared
(e) share\\\\
14. (a) own (b) govern
(c) regulations (d) ran
(e) supervised\\\\
15. (a) Such (b) Partly
(c) Whole (d) Most
(e) Entire\\\\
16.(a) visit (b) help
(c) borrow (d) loan
(e) advice\\\\
17.(a) include (b) accounted
(c) fulfilled (d) achieved
(e) taking\\\\
18.(a) branches (b) all
(c) operating (d) staff
(e) factory\\\\
19.(a) thinking (b)known
(c) creating (d) cross
(e) develop\\\\
20. (a) make (b)release
(c) boosting (d)stall
(e) revitalise\\\\

      \textbf{Answers} - \\\\
11. (b) Enjoy\\\\
12. (d) Provide\\\\
13. (e) share (?)\\\\
14. (e) Supervised\\\\
15. (d) Most\\\\
16. (c) Borrow\\\\
17. (b) Accounted\\\\
18. (a) Branches\\\\
19. (b) Known (?)\\\\
20. (e) Revitalise\\\\

  \chapter{Day 2 - 05/01/2024}
    Pick me up strategies discussed.

    \section{Abbreviations}
      Abbreviations are shortened forms of words or phrases. They are created by taking the initial letters or syllables of a phrase and condensing it into a shorter form, usually preserving the main components of the original term. Abbreviations are commonly used to save space, time, or effort in writing or speech.\\\\
      
     \underline{An Assorted list of Abbreviations category wise is given in the Material Folder.}\\\\
    
    \section{F.R.I.E.N.D.S?}
      A video of the TV show \textbf{F.R.I.E.N.D.S} was opened on \href{https://youtube.com}{Youtube}\\
      It is the scene of joey writing letter of recomendation for \textbf{Monica and Chandler} who want to adopt a baby. He starts off childish, then gets help with ross. What happens next is shocking!!!\\\\
      
    \section{Aftermath}
      New Topic of spelling error correction started. Mistakes have been highlighted in bold. \\\\
      Recently, researchers who study chimpanzees have come to the \textbf{suprising} conclusion that groups of chimpanzees have
their own traditions that can be \textbf{past} on to new generations of chimps. The chimps do not \textbf{aquire} these traditions by
instinct; instead, they learn them from other chimps. When a scientific journal published \textbf{analysises} of chimpanzee
behavior, the author revealed that the every day actions of chimpanzees in \textbf{seperate} areas differ in significant ways, even
when the groups belong to the same subspecies. For instance, in one West African group, the chimps are often seen \textbf{puting}
a nut on a stone and using another \textbf{peice} of stone to crack the nut open, a kind of behavior never observed in other groups
of chimpanzees. \textbf{Sceintists} have also observed the chimps teaching \textbf{there} young the nut opening method, and chimps in
other places that crack nuts \textbf{differentally} teach their young \textbf{they’re} own way. Researchers have \textbf{therefor} concluded that
chimpanzees have local traditions.\\
Frans de Waal, who has been \textbf{studing} primates, wrote a book \textbf{makeing} the \textbf{arguement} that these learned behaviors
should be considered kinds of culture. The word culture has \textbf{traditionly} been used to describe human \textbf{behavier}, but may be,
he says, a new definition is needed. Considering this \textbf{startlingly-new} theory of chimpanzee “culture,” some researchers
think that humans now have an \textbf{un-deniable} obligation to protect the lives of all remaining wild chimpanzees rather than
\textbf{zeroeing} in on just a few of the \textbf{threatenned} animals. The \textbf{lost} of a single group of wild chimpanzees would, they say,
destroy something irreplacable, a unique culture with its own traditions and way of life.\\

  \chapter{Day 3 - 08/01/2024}% (fold)
  \section{Changing of Sentences}
    \textbf{Statement} - \\
    \par Making Sentences:
Write a meaningful sentence about your life that gives some specific information. For example: \\
1. When I was walking back home, I saw a little girl chasing a butterfly.\\
2. It is my dream to become the richest man in the country.\\

Now, look for the meaning of different words and find their synonyms, antonyms or any other related words. 

Try to replace the words and change the meaning of the sentence. For the given example, here are the modified sentences:\\
1. While I was accompanying promote home, I spied a little girl tracking a butterfly.\\
2. It is my hallucination to become the most expensive fellow in the nation.\\

Do you see how the modification of words alters the meaning of the sentence? Similar words have specific meanings, and choosing \textbf{the write} (AC2) words makes communication effective.

Instructions: \\
1. Avoid simple sentences like - I like to play basketball, I take a bus after college etc.\\
2. Change at least 2 words in the sentence.\\

Additional Activity 1 (AC1):\\
Make a sentence using as many modified words as possible. Write this below the two sentences.\\
Additional Activity 2 (AC2):\\
Find an incorrectly used word in the description of this Class Activity. \\\\

    \textbf{Answer} - 
    \begin{itemize}
      \item  \textbf{Sentence} - I was at a party when I met a peculiar stranger who showed me some spectacular party tricks. 
      \item \textbf{Changed} -  I was at a party when I met a funny stranger who manifested his talents in the form of some truly spectacular party tricks.
      \item \textbf{New Changed Sentence (AC1)} - I was attending to a social gathering where I assembled a funny alien the likes of whom manifested his talents in the form of some truly spectacular gathering ploys. 
    \end{itemize}
  
  % chapter Day 3 -  (end)

  \chapter{Day 4 - 11/01/2024} Today we study collocations and go over some spelling rules. Also Read a \href{https://drive.google.com/file/d/1--IABOxrbkADAgSGHDRYENKqqDqNwP_j/view?usp=sharing}{\textbf{Poem}}\\
    \section{Spelling rules}
      Studies some \href{https://www.awelu.lu.se/language/spelling/english-spelling-rules/?authuser=1}{Spelling Rules}\\
      The exercises for this was done in Previous classes. \\\\
      Syllabus Topics - 
      \begin{itemize}
        \item Spelling adverbs
        \item Spelling comparatives and superlatives
        \item Spelling: Double consonants
        \item Spelling plurals
        \item Spelling: Silent letters
        \item Spelling similar sounds
        \item Spelling and unstressed syllables
        \item Spelling word endings
        \item Spelling verb inflections
      \end{itemize}
      
  \chapter{Day 5 - 12/01/2024}
    \section{Collocations}
      \href{https://www.englishclub.com/vocabulary/collocations.php}{Material Link} - Follow this for more Info.
      \par It seems like there might be a typo in your question. If you meant "collocations," collocations refer to the habitual juxtaposition of a particular word with another word or words with a frequency greater than chance. In linguistics, collocation is the habitual juxtaposition of a particular word with another word or words with a frequency greater than chance. These combinations of words can be commonly found together in spoken or written language.
      \par For example, in English, we often use the collocation "strong coffee" instead of "powerful coffee" or "vigorous coffee." Collocations are an essential aspect of language use, and understanding them helps in achieving natural and fluent expression. They contribute to the richness and nuance of language.\\\\
      
      Class Quizzes - \\
      \par \href{https://www.ieltsbuddy.com/collocation-quiz.html?authuser=1}{\textbf{QUIZ 1}} - IELTS Buddy
      \par \href{https://www.ldoceonline.com/quiz/section-collocations/?authuser=1}{\textbf{QUIZ 2}} - Longman\\\\


      \par \textbf{Answers Q1} -  
      \begin{enumerate} 
        \item have 
        \item have 
        \item makes
        \item have 
        \item made 
        \item have 
        \item makes
        \item have 
        \item do 
        \item have 
      \end{enumerate}

      \newpage 

      \par \textbf{Answers Q2} -  (Less common part)
      \begin{enumerate}
        \item Misunderstanding
        \item Devise
        \item Empire 
        \item Make-up
        \item Earthquake
        \item Stranger
      \end{enumerate}

      \par \textbf{Answers Q2} - (Common Part)
      \begin{enumerate}
        \item Variety 
        \item Bed 
        \item Effective 
        \item Mistake 
        \item Deal
        \item Stage
      \end{enumerate}

      \par Website - \href{https://englishclub.com/ref/Collocations}{\textbf{Collocations}}\\
      \par \underline{Me to my kids - You were a Sexual Oversight}
  
  \chapter{Day 6 - 15/01/2024}
    First we discussed the common misused words and then we started discussing the quizzes and their answers. Today I will not mark the answers.\\\\ 
    \section{Commonly Misused Words}
     hello fremds! \\\\ 
      \textbf{Materials} :\\
        \href{https://www.touro.edu/departments/writing-center/tutorials/commonly-misused-words/?authuser=1}{Touro}\\
        \href{https://webapps.towson.edu/ows/exercises/posttest2.aspx?authuser=1}{WebApps}\\
        \href{https://drive.google.com/file/d/1-yyI_RZh5uIYTxD-jouX7BSp531Qydqm/view?usp=sharing}{PdfFile}\\\\
      \textbf{Quizzes} : \\
        \href{https://www.niu.edu/writingtutorial/grammar/quizzes/ConfusedWords.htm?authuser=1}{Quiz1}- 26 Qs\\
        \href{https://public.wsu.edu/~campbelld/amlit/quiz/usage.htm?authuser=1}{Quiz2}\\\\
      \textbf{Answers} : \\
        Quiz 1 - Accede, Except, Humane, Interstate, Capital, Complemented, Borne, All Ready, Principal, Role, Affected, Advise, Morale, Stationery, Simple, Loath, Discreet/Envelope, Site, Personnel, Implied(?Inferred), Led, Lay/Lain, Uninterested, As if, Fewer, An Oral\\\\
  
  \chapter{Day 7 - 18/01/2024} % (fold)
  \label{chap:Day 7 - 18/01/2024}

    \section{One Word Substitution}
      \underline{The PDF resources uploaded to Materials folder}\\\\

      \par \textbf{Quiz} - \href{https://careericons.com/general-english-mcq/one-word-substitution/section-1-question-answer/154-1/}{CareerIcons}\\\\

      \par \textbf{Answers} - (Section 1) 
        \begin{enumerate}
          \item Inflammatory
          \item Cartoon
          \item Euthanasia
          \item Subsistence
          \item Compatriots
        \end{enumerate}
  
  % chapter Day 7 - 18/01/2024 (end)


  \chapter{Day 8 - 19/01/2024} % (fold)
  \label{chap:Day 8 - 19/01/2024}
    Abbreviation and Foreign words!
    \section{Revision of Foreign Words!}
       Ma'am revised the topic and posted a google form exercise. 
  % chapter Day 8 - 19/01/2024 (end)

  \chapter{Day 9 - 25/01/2024} % (fold)
  \label{chap:Day 9 - 25/01/2024}
    Abbreviations practised again. 
    Also, idk man go study. 

  \chapter{Day 10 - 29/01/2024} % (fold)
  \label{chap:Day 10 - 29/01/2024}
    \section{Tenses?} % (fold)
    \label{sec:Tenses?}
    
      A youtube \href{https://www.youtube.com/watch?v=54prMaPn5Ls&authuser=1}{video} was played explaining tenses. 
      \begin{itemize}
        \item What are tenses?
        \item Grammatical Aspects \begin{itemize}
        \item \begin{itemize}
            \item Simple
            \item Continuous
            \item Perfect
            \item Perfect Continuous
            \item Progressive and Perfect Progressive
          \end{itemize}
        \end{itemize}
      \end{itemize}
    % section Tenses? (end)
      Aspects tell us about the status of the action at a particular moment.\\\\
      \textbf{HW} - Revisit at home.\\\\
    
    \section{Exercises as Usual!}
      \href{https://www.english-grammar.at/online_exercises/tenses/tenses_index.htm}{\textbf{QUIZ}}
  
  % chapter Day 10 - 29/01/2024 (end)
  
  % chapter Day 9 - 25/01/2024 (end)

\end{document} 
